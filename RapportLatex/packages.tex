% Dictionnaire français, pour caractères spéciaux, tirets, caractères accentués
\usepackage[french]{babel}
\usepackage[utf8]{inputenc}
%Toujours plus d'accents
\usepackage[T1]{fontenc}

\usepackage{float} %utile pour ancrer les images (avec [H])

% Pour les tableaux
\usepackage[table]{xcolor}
\usepackage{array}
\usepackage{tabularx}
\usepackage{multirow}
\usepackage{longtable}

\usepackage{indentfirst}

\usepackage[T1]{fontenc} % Output font encoding for international characters

\usepackage{mathpazo} % Use the Palatino font by default

\usepackage[backend=bibtex,style=authoryear,natbib=true]{biblatex} % Use the bibtex backend with the authoryear citation style (which resembles APA)

\author{Damien \textsc{Galbrun}}

%hauteur entre deux lignes
\baselineskip 200cm
%hauteur entre deux paragraphes
\parskip 2mm
%longueur d'indentation
\parindent 2mm
%(on utilise \indent et \noindent sinon)

%Gérer ses marges
%Facilement
\usepackage[margin=2cm]{geometry}

%Avoir des entêtes et pieds de page stylés
\usepackage{fancyhdr}
\pagestyle{fancy}

% Liens dans le document
\usepackage{hyperref}  
% Légendes dans les environnements "figure" et "float"
\usepackage{subcaption}
%La base pour faire des figures juste
\usepackage{graphicx}
\usepackage[export]{adjustbox}
\usepackage{wrapfig}
%Trucs utiles pour les maths
\usepackage{amsmath}